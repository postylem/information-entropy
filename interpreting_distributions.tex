%!TEX program = xelatex
\documentclass{article}

\usepackage{amsfonts,amssymb,amsthm}
\usepackage{enumitem,booktabs}
\usepackage[margin=1in]{geometry}
\usepackage{parskip}
\usepackage{fontspec}
\setmainfont{Linux Libertine}

\usepackage[%
natbib=true,backend=biber,sorting=nymdt,%
citestyle=authoryear,bibstyle=authoryear,%
url=false,doi=false,isbn=false%
]{biblatex}
\addbibresource{/Users/j/Library/texmf/bibtex/bib/all.bib}

% jbib_links: makes nice links, and defines nymdt sorting.
%   Imports mathtools, xcolor, hyperref, cleveref.
\input{/Users/j/Library/texmf/custom/jbib_links.tex}

% jformat: general formatting things.
%   Imports packages contour, ulem
\input{/Users/j/Library/texmf/custom/jformat.tex}

% kast: a nice box environment:
\input{/Users/j/Library/texmf/custom/kast.tex}

% IT operators: for expectation \E{}, entropy \H{}, MI \I{} etc.
%\input{/Users/j/Library/texmf/custom/information-theory-operators.tex}

\newtheorem*{theorem}{Theorem}
\theoremstyle{definition}
\newtheorem*{definition}{Definition}
\newcommand\key[1]{\myuline{#1}}
%\usepackage{gb4e}\noautomath% removes incompatibilities
\begin{document}

\title{Binomial to Gamma}
\author{}%\author{Jacob Louis Hoover}
\date{}

\maketitle
\begin{kast}{Binomial}
\begin{equation}
S \sim \mathrm{Binomial}(n,p)
\end{equation}
Interpretation: $S\in \{0,\ldots,n\}$ is the number of successes in $n$ repeated trials,
each with probability $p$ of success.
\begin{itemize}
  \item If $n=1$, it is called $\mathrm{Bernoulli}(p)$.
\end{itemize}
\end{kast}

If we think of the $n$ trials happening in a fixed period of time, and divide
the period infinitesimally, so instead of $n,p$, we have a \key{rate},
$\lambda$, the expected number of successes, we get the Poisson.

\begin{kast}{Poisson}
  \begin{equation}
    S \sim \mathrm{Poisson}(\lambda)
  \end{equation}
  Interpretation: $S\in\{1,\ldots,\infty\}$ is the number of successes in a
  fixed period, given constant rate $\lambda$.
\end{kast}

If instead of being interested in the \emph{number} of successes, we're
interested in the \emph{time between successes}, we get the Exponential
distribution (or more generally, the Gamma distribution).

\begin{kast}{Exponential}
  \begin{equation}
    S \sim \mathrm{Exponential}(\lambda)
  \end{equation}
  Interpretation: $S\in \mathbb{R} \ge 0$ is the time to wait before the first
  success in a Poisson process with rate $\lambda$.
  \begin{itemize}
    \item sometimes parametrized in terms of the mean $\frac{1}{\lambda}$ (units
      of time), rather than rate $\lambda$ (units inverse time).
    \item if rate is not constant, but is proportional a power of time, see
      Weibull distribution (common for time-to-failure interpretation).
  \end{itemize}
\end{kast}

\begin{kast}{Gamma}
  \begin{equation}
    S \sim \mathrm{Gamma}(k,\lambda)
  \end{equation}
  Interpretation: $S\in \mathbb{R} \ge 0$ is the time it takes to have $k$
  successes in a Poisson process with rate $\lambda$. 

  The `shape' parameter $k$ can be any positive real.
  \begin{itemize}
    \item if $k$ is restricted to be an integer, then it is called
      $\mathrm{Erlang}(k,\lambda)$.
    \item if $k=1$, it's $\mathrm{Exponential}(\lambda)$.
  \end{itemize}
\end{kast}

\nocite{leemis.l:2008,crooks.g:2019}
\printbibliography[title=References]{}
\end{document}
