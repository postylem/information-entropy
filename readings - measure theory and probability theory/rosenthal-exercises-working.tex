%!TEX program = xelatex
\documentclass[]{scrartcl}
\usepackage{amsmath,amsfonts,amssymb,mathtools}
\usepackage{enumitem,booktabs}
\setlist[enumerate]{label=\normalfont\bfseries(\alph*)}
\usepackage[margin=1in]{geometry}
\usepackage[no-math]{fontspec}
\setmainfont{Palatino}
\setsansfont{Palatino}
\usepackage[%
natbib=true,backend=biber,sorting=nymdt,%
citestyle=authoryear,bibstyle=authoryear,%
url=false,doi=false,isbn=false%
]{biblatex}
\addbibresource{/Users/j/Library/texmf/bibtex/bib/all-biblatex.bib}
% import xcolor, imports and configs hyperref, defines nymdt sorting:
\input{/Users/j/Library/texmf/custom/jbib_links.tex}
\input{/Users/j/Library/texmf/custom/kast.tex}
\usepackage{cleveref} % must come after importing hyperref
\input{/Users/j/Library/texmf/custom/jformat.tex} % general formatting things
\input{/Users/j/Library/texmf/custom/information-theory-operators.tex}

\usepackage[amsthm]{ntheorem}
% \theoremstyle{plain}%default
\newtheorem*{theorem}{Theorem}
\theoremstyle{break}
\newtheorem*{exercise}{Exercise}

\theoremstyle{definition}
\newtheorem*{definition}{Def}
\newtheorem{exmp}{Example}[section]

\theoremstyle{break}\theorembodyfont{\normalfont}\theoremheaderfont{\itshape}
\newtheorem*{solution}{Solution}

\newcommand\key\myuline%
\renewcommand\right\mright%
\renewcommand\left\mleft%

\begin{document}

\title{Solutions to Odd-numbered Exercises}
\subtitle{of \citetitle{rosenthal.j:2006} \citep{rosenthal.j:2006}}
\author{Jacob Louis Hoover}
\date{}

\maketitle

\section*{Chapter 2: Probability Triples}

\begin{exercise}[2.2.3]
  Prove that the collection
  \(\mathcal{J}=\{\text{all intervals contained in }[0,1]\}\)
  is a \emph{semialgebra} of subsets of $\Omega$, meaning that it contains
  $\emptyset$ and $\Omega$, it is closed under finite intersection, and the
  complement of any element of $\mathcal{J}$ is equal to a finite disjoint
  union of elements of $\mathcal{J}$.
  \begin{solution}
  Show it obeys the requirements
    \begin{itemize}[noitemsep]
      \item by definition $\mathcal{J}$ contains $\emptyset$ and $\Omega$
      \item the intersection of any finite number of intervals is an interval
      \item complement of any interval can be made by taking the union of disjoint
        intervals (only at most two are needed: before and after).
    \end{itemize}
  \end{solution}
\end{exercise}

\begin{exercise}[2.2.5]
  \begin{enumerate}
    \item Prove that $\mathcal{B}_0 = \{\text{all finite unions of elements of }
      \mathcal{J}\}$ is an \emph{algebra} (or \emph{field}) of subsets of
      $\Omega$, meaning that it contains $\Omega$ and $\emptyset$, and is closed
      under formation of complements and of \emph{finite} unions and
      intersections.
    \item Prove thet $\mathcal{B}_0$ is \myuline{not} a $\sigma$-algebra.
  \end{enumerate}

  \begin{solution}
    \begin{enumerate}
      \item
      \begin{itemize}
        \item by definition $\mathcal{B}_0$ it contains $\emptyset$ and $\Omega$
        \item it is closed under formation of complements because the complement
          of any element of $\mathcal{J}$ is a finite disjoint union of elements
          of $\mathcal{J}$, which is in $\mathcal{B}_0$ by definition
        \item likewise it is closed under finite unions, since any finite union
          of elements of $\mathcal{B}_0$ is a finite union of elements of
          $\mathcal{J}$.
        \item for any finite set $\{A_i\}_i$ of elements of $\mathcal{B}_0$,
          the intersection $\bigcap A_i= {(\bigcup A_i^c)}^c \in\mathcal{B}_0$ 
          by closure under complements and finite unions.
      \end{itemize}
    \item Assume $\mathcal{B}$ is a $\sigma$-algebra. Then it is closed under 
      formation of countable unions. Take the set
      $N=\bigcup_{n\in\mathbb{N}}(\frac{1}{2^{n+1}},\frac{1}{2^n})$, which
      is a countable union of intervals in $[0,1]$, so it is in $\mathcal{B}$.
      However, the set $N$ consists of an infinite number of disjoint intervals,
      so it cannot be constructed from finite union of intervals,
      thus is not in $\mathcal{B}_0$. Contradiction.
    \end{enumerate}
  \end{solution}
\end{exercise}


\printbibliography[title=References]{}
\end{document}
